\documentclass[letterpaper, 12pt]{article}

\usepackage[margin=1in]{geometry}

\begin{document}

    \thispagestyle{empty}

    \begin{center}

        \textbf{\large{Right Leadership}}

    \end{center}

    \noindent \textbf{Introduction}

    Good evening, I'm thankful for the opportunity to be here tonight.
    Ed approached me a few months ago and asked if I could do a
    three-part series during the next three Sunday evenings, and one of
    the first things that came to mind was that we have several new
    members and some visitors and regular non-members as well. I've been
    in that kind of situation before quite frequently, what with moving
    around in the military and all, and it has always struck me that
    there is a surprising lack of teaching and resources available to
    the average person on how to handle that. So the series I want to do
    over the next three weeks covers the question of ``What to Look For
    in a Church'', and I hope this series will be beneficial not only to
    those who are new to our church, but also to all of us as a good
    reminder of some things that we can use to do some self-assessment
    of our church body as a whole as well.

    Almost exactly one year ago today, I was just beginning to teach
    through our Introduction to Church History class, and towards the
    beginning I offered a description of what constitutes a solid,
    Biblical, Godly church. I posited that a good church has three
    characteristics: Right Leadership, Right Doctrine, and Right
    Practice. What I want to do over the next three Sunday evenings is
    expand on each of those, and show how the Bible exposits those as
    fundamental tenets of a church that pleases God. So we'll take one
    of those each week, and tonight we'll start with Right Leadership.

    Leadership is a tricky thing to understand. It seems natural that
    there is a ``two sides of the coin'' effect with the ideas of
    leaders and followers. On one side of the coin, you have leaders; on
    the other side, you have followers. That makes sense, doesn't it?
    But there is a thought that has developed today that because of that
    partnership between leaders and followers, then there is equal
    weight and emphasis on both leaders and followers. For example, I
    know of one pastor that I highly recommend who, whenever anyone
    brings up issues with the state of leadership in the Church today,
    he will always turn around and point to the followers as the root of
    the problem.

    I don't think that's quite accurate. I think the Bible puts a
    premium on leadership as having the responsibility to lead a local
    church, and whenever you see a church that goes astray, yes, the
    followers bear some responsibility for that, but I think that
    ultimately the burden and root of the issues lies with leadership.
    Let me give you just a couple texts that I think illustrate this.
    
    First, in Hebrews 13:17, Christians are commanded to ``Obey your
    leaders and submit to them---for they keep watch over your souls as
    those who will give an account\ldots'' Pastors and leaders in the
    church will give an account not just of themselves, but also of the
    flock they pastor. Think of it this way: when a pastor stands before
    the Lord, imagine the Lord saying, ``I gave you a charge to keep
    watch over Bob's soul in your church, and yet Bob persisted in a
    life of sin. Did you do anything to address it? Did you follow my
    command to preach the Word? Did you warn him? Did you even know of
    his life of sin?'' And there are many pastors today who will be able
    to do nothing but hang their heads in shame, because they did not
    keep watch over the souls of their flock, and so they will fail that
    account when they stand before the Lord.

    Another text that shows an emphasis on leadership is found in
    Ephesians 4:11--12. Paul writes, ``And He Himself gave some as
    apostles, and some as prophets, and some as evangelists, and some as
    pastors and teachers, for the equipping of the saints for the work
    of service, to the building up of the body of Christ\ldots'' Here we
    see that saints are to be equipped for ``the work of service'' which
    is defined as ''the building up of the body of Christ'', but notice
    how this equipping is accomplished. It is accomplished through the
    work of, among others, ``pastors and teachers''. And notice that
    pastors and teachers are singled out as being gifts that have been
    given by the Lord Jesus Christ Himself.

    So the Bible does in fact put a premium on leadership within the
    church, and the first thing to note for any particular church is the
    state of their leadership. So what are we to make of leadership
    within the church? Is there any Biblical guidance available to us
    for how to judge church leadership? Does the Bible say anything to
    us about what church leadership looks like, who should be a church
    leader, and what do church leaders do? 

    And of course, the answer is yes. The Bible says a great deal about
    leadership, and so tonight we're going to examine that. \\

    \noindent \textbf{Biblical Eldership} \\

    One of the best, most comprehensive volumes on the subject of church
    leadership is Alexander Strauch's \emph{Biblical Eldership}. It's
    the best resource I can recommend today, and it's not even close. In
    it, Strauch exposits every single passage in the Bible that touches
    on leadership in the church, and one of the main topics he addresses
    is that of elder qualifications. Who is qualified to be a pastor in
    a church? Let's look at those briefly before we get into our passage
    for tonight: \\

    \noindent \emph{Spirit-Given Motivation for the Task} \\

    The first qualification we find in Scripture is what is often termed
    a ``pastoral calling''. We often say that a man who wishes to be a
    pastor must be ``called'' by God. I think Strauch articulates it
    better by pointing to 1 Timothy 3:1 and identifying this call as a
    ``spirit-given motivation for the task''. In other words, God
    doesn't call pastors the same way as He did Old Testament prophets,
    for example. Rather, God gives men a desire and a burden to
    accomplish the task of elder, and part of working at the task
    includes being recognized in the office of pastor. We don't have
    time to exposit through 1 Timothy 3:1 completely, but I can safely
    say that what Paul has in view here when he writes ``If a man
    desires the office of elder, he desires a noble thing,'' is that the
    desire is for the task, not the status. And so the first
    qualification is that man who would lead a church must have a burden
    for serving the Lord in that way. \\

    \noindent \emph{Exemplary Character Qualities} \\

    The next set of qualifications we find in both 1 Timothy 3 and the
    parallel passage in Titus 1 is that leaders in the church must
    possess exemplary character qualities. That is, they must be of
    impeccable moral character such that they reflect God's holiness and
    do not bring dishonor upon the Lord or the church. And just looking
    at these lists for a moment, we see that the first, over-arching
    qualification is to be ``above reproach''. The idea of being above
    reproach functions in two ways: it acts as a general catch-all to
    describe someone of impeccable character, and it also acts as an
    umbrella term to capture all the rest of the character qualities
    that follow in these passages. The idea of above reproach means that
    there's not always something going on with a leadeer, there's not
    any unresolved issues, and that any issue must be fully examined and
    understood before rendering a leader unfit for office.

    Following that over-arching qualification, Paul gives us a familiar
    list of moral imperatives that should describe the manner of life of
    a church leader. In 1 Timothy 3, he is to be a one-woman man,
    temperate, self-controlled, respectable, hospitable, not a drunkard,
    not violent or argumentative, considerate, peaceable, not greedy. He
    is to manage his own household as an indication of fitness for
    leadership position. He must be mature in faith. He must possess a
    good reputation with outsiders. Then, in Titus 1, we see a similar
    list, and Paul includes a prohibition against being quick-tempered,
    and a requirement to love what is good, to be righteous and holy.

    Now, I would love to take several weeks and exposit through each of
    these words, but what I want to highlight two things about these
    lists. First, both of these lists of pastoral qualifications are
    written in the present-tense. Meaning, they are qualities that a man
    must possess \emph{now}. A man, twenty years ago in his youth, may
    not have met these qualifications, and that doesn't matter. The
    issue is how is he living his life now? Second, don't think for a
    moment that just because this is a list for qualifications of pastor
    that this list doesn't apply to you. We're going to see that in a
    moment, but suffice to say that the reason these character
    attributes are listed is so that the man of God will be qualified
    for leadership, and then we as those who submit to leadership will
    be able to watch those qualities lived out in his life and then
    develop and emulate them in us! In other words, if you want to know
    what being self-controlled looks like, what being peaceable looks,
    what being righteous looks like, examine the lives of your leaders
    and imitate them in those things. \\

    \noindent \emph{Made Skillful for the Job} \\

    The final category of qualification that Strauch makes is to
    highlight that a pastor is to be gifted for the job. Specifically,
    the pastor is to have the ability to teach sound doctrine in
    accordance with God's Word and explain and apply the Scriptures to
    the flock so that they will know and understand what God's
    prescriptions, precepts, principles, and prohibitions are and how to
    live those out in our lives. Both 1 Timothy 3 and Titus 1 include a
    requirement to ``be able to teach'', and Titus 1 further explains
    that one of the reasons for this is to be able to refute false
    teaching and ``reprove those who contradict'' it. Not everyone is a
    teacher, and those that are teachers and leaders in a church are
    held to a higher standard (James 3:1), and so Paul lays that out as
    a requirement for leadership in the church. \\

    \noindent \textbf{1 Peter 5:1--3} \\

    So, all of that is introduction, and we could months just poring
    over all of those things, but I want now to take a step back and
    look at the bigger picture. What is going on with all of these
    things, how do they fit into the larger design that God has for
    building His Church? And the key question is this: What is it that
    elders are supposed to do? What should we expect elders to do in the
    performance of their job? That's really where the rubber meets the
    road, because this is where right leadership is distinguished from
    bad leadership.

    Alexander Strauch points out several sweeping phrases that describe
    what elders are expected to do as the leaders of the church. Using
    Biblical descrptions, He points out that elders are to feed the
    flock. Elders are to guard the flock. Elders are to lead the flock.
    We might expect elders to conduct some sort of visitation or
    ministry to those who are chronically ill in the congregation. All
    these things are things we might consider as part of the job
    description of pastor.

    But there is one passage that I find in Scripture that really sums
    up the totality of pastoral ministry. All other functions or
    ministry imperatives we find in Scripture can really be fit under
    this one particular command we find in this passage. And that is
    Peter's command in 1 Peter 5:1--3 to ``shepherd the flock.'' So turn
    with me there now and we're going to spend the rest of the time
    expositing our way through this passage to see the essence of what
    Right Leadership will look like in a church.

    $ $\\

    \noindent \textbf{Conclusion} \\

\end{document}

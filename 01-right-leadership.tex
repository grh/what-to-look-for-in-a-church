\documentclass[letterpaper, 12pt]{article}

\usepackage[margin=1in]{geometry}

\begin{document}

    \thispagestyle{empty}

    \begin{center}

        \textbf{\large{Right Leadership}}

    \end{center}

    \noindent \textbf{Introduction}

    Good evening, I'm thankful for the opportunity to be here tonight.
    Ed approached me a few months ago and asked if I could do a
    three-part series during the next three Sunday evenings, and one of
    the first things that came to mind was that we have several new
    members and some visitors and regular non-members as well. I've been
    in that kind of situation before quite frequently, what with moving
    around in the military and all, and it has always struck me that
    there is a surprising lack of teaching and resources available to
    the average person on how to handle that. So the series I want to do
    over the next three weeks covers the question of ``What to Look For
    in a Church'', and I hope this series will be beneficial not only to
    those who are new to our church, but also to all of us as a good
    reminder of some things that we can use to do some self-assessment
    of our church body as a whole as well.

    Almost exactly one year ago today, I was just beginning to teach
    through our Introduction to Church History class, and towards the
    beginning I offered a description of what constitutes a solid,
    Biblical, Godly church. I posited that a good church has three
    characteristics: Right Leadership, Right Doctrine, and Right
    Practice. What I want to do over the next three Sunday evenings is
    expand on each of those, and show how the Bible exposits those as
    fundamental tenets of a church that pleases God. So we'll take one
    of those each week, and tonight we'll start with Right Leadership.

    Leadership is a tricky thing to understand. It seems natural that
    there is a ``two sides of the coin'' effect with the ideas of
    leaders and followers. On one side of the coin, you have leaders; on
    the other side, you have followers. That makes sense, doesn't it?
    But there is a thought that has developed today that because of that
    partnership between leaders and followers, then there is equal
    weight and emphasis on both leaders and followers. For example, I
    know of one pastor that I highly recommend who, whenever anyone
    brings up issues with the state of leadership in the Church today,
    he will always turn around and point to the followers as the root of
    the problem.

    I don't think that's quite accurate. I think the Bible puts a
    premium on leadership as having the responsibility to lead a local
    church, and whenever you see a church that goes astray, yes, the
    followers bear some responsibility for that, but I think that
    ultimately the burden and root of the issues lies with leadership.
    Let me give you just a couple texts that I think illustrate this.
    
    First, in Hebrews 13:17, Christians are commanded to ``Obey your
    leaders and submit to them---for they keep watch over your souls as
    those who will give an account\ldots'' Pastors and leaders in the
    church will give an account not just of themselves, but also of the
    flock they pastor. Think of it this way: when a pastor stands before
    the Lord, imagine the Lord saying, ``I gave you a charge to keep
    watch over Bob's soul in your church, and yet Bob persisted in a
    life of sin. Did you do anything to address it? Did you follow my
    command to preach the Word? Did you warn him? Did you even know of
    his life of sin?'' And there are many pastors today who will be able
    to do nothing but hang their heads in shame, because they did not
    keep watch over the souls of their flock, and so they will fail that
    account when they stand before the Lord.

    Another text that shows an emphasis on leadership is found in
    Ephesians 4:11--12. Paul writes, ``And He Himself gave some as
    apostles, and some as prophets, and some as evangelists, and some as
    pastors and teachers, for the equipping of the saints for the work
    of service, to the building up of the body of Christ\ldots'' Here we
    see that saints are to be equipped for ``the work of service'' which
    is defined as ''the building up of the body of Christ'', but notice
    how this equipping is accomplished. It is accomplished through the
    work of, among others, ``pastors and teachers''. And notice that
    pastors and teachers are singled out as being gifts that have been
    given by the Lord Jesus Christ Himself.

    So the Bible does in fact put a premium on leadership within the
    church, and the first thing to note for any particular church is the
    state of their leadership. So what are we to make of leadership
    within the church? Is there any Biblical guidance available to us
    for how to judge church leadership? Does the Bible say anything to
    us about what church leadership looks like, who should be a church
    leader, and what do church leaders do? 

    And of course, the answer is yes. The Bible says a great deal about
    leadership, and so tonight we're going to examine that.

    \noindent \textbf{Biblical Eldership}

    One of the best, most comprehensive volumes on the subject of church
    leadership is Alexander Strauch's \emph{Biblical Eldership}. It's
    the best resource I can recommend today, and it's not even close. In
    it, Strauch exposits every single passage in the Bible that touches
    on leadership in the church, and one of the main topics he addresses
    is that of elder qualifications. Who is qualified to be a pastor in
    a church? Let's look at those briefly before we get into our passage
    for tonight:

    \noindent \emph{Spirit-Given Motivation for the Task}

    The first qualification we find in Scripture is what is often termed
    a ``pastoral calling''. We often say that a man who wishes to be a
    pastor must be ``called'' by God. I think Strauch articulates it
    better by pointing to 1 Timothy 3:1 and identifying this call as a
    ``spirit-given motivation for the task''. In other words, God
    doesn't call pastors the same way as He did Old Testament prophets,
    for example. Rather, God gives men a desire and a burden to
    accomplish the task of elder. We don't have time to exposit through
    1 Timothy 3:1 completely, but I can safely say that what Paul has in
    view here when he writes ``If a man desires the office of elder, he
    desires a noble thing,'' is that the desire is for the task, not the
    status. And so the first qualification is that man who would lead a
    church must have a burden for serving the Lord in that way.

    \noindent \emph{Exemplary Character Qualities}

    \noindent \emph{Made Skillful for the Job}

    \noindent \textbf{1 Peter 5:1--3}

    \noindent \textbf{Conclusion}

\end{document}

\documentclass[letterpaper, 12pt]{article}

\usepackage[margin=1in]{geometry}

\begin{document}

    \thispagestyle{empty}

    \begin{center}

        \textbf{\large{What To Look For In A Church: Right Leadership}}

    \end{center}

    \noindent \textbf{Introduction} \\

    Good evening, I'm thankful for the opportunity to be here tonight.
    Ed approached me a few months ago and asked if I could do a
    three-part series during the next three Sunday evenings while he and
    his family are out of town, and one of the first things that came to
    mind was that we have several new members and some visitors and
    regular non-members as well. I've been in that kind of situation
    before quite frequently, what with moving around in the military and
    all, and it has always struck me that there is a surprising lack of
    teaching and resources available to the average person on how to
    handle that. So the series I want to do over the next three weeks
    covers the question of ``What to Look For in a Church'', and I have
    three aims for what I hope will be beneficial in this series. 

    First, I hope this provides those who are new to our church some
    solid ground on which to guide or confirm their decision and to
    encourage them to church membership if you haven't done so already.
    Second, not only for those who are new to the church, but also to
    all of us as a good reminder of some things that we can use to do
    some self-assessment of our church body as a whole as well. 

    But there one more goal I have for this series. And in order to
    understand that, you need to understand that there are a ton of
    churches in the area that are either ignorant of what the Bible says
    about churches, or they're just lax / fearful of driving people
    away, or, and this is what I suspect is most of these churches, they
    would actually deny the things we're going to go over. Now, each of
    us interacts with people in those churches almost every day. Maybe
    you work with some of the people in these churches, maybe you're
    friends with some of them, maybe they're even some of your
    relatives. And you need to realize that they're not getting what
    they need to be nourished spiritually, and in some cases they're not
    even getting what they need to know in order to be saved.

    And so the third goal I hope for this series is that it will give
    you some Biblical tools to talk to people that are in these churches
    that know that something is wrong with where they're at, and even
    invite them here where at least the truth of the Bible is made
    paramount in obedience to Christ. In other words, I hope you can use
    this series to launch a rescue mission.
    
    Almost exactly one year ago today, I was just beginning to teach
    through our Introduction to Church History class, and towards the
    beginning I offered a description of what constitutes a solid,
    Biblical, Godly church. I posited that a good church has three
    characteristics: Right Leadership, Right Doctrine, and Right
    Practice. What I want to do over the next three Sunday evenings is
    expand on each of those, and show how the Bible exposits those as
    fundamental tenets of a church that pleases God. So we'll take one
    of those each week, and tonight we'll start with Right Leadership.

    Like most doctrines, today's church is rife with outrageous error on
    the subject of church leadership. There is a crisis of leadership in
    churches across the country today. There are two main fights going
    on right now that most churches are losing: first is the CEO model
    versus servant leadership. No less than heretic Andy Stanley has
    said this about church leadership. He was asked should we stop
    referring to pastors as shepherds, and he responded, 

    \begin{quote}

            ``Absolutely.  That word needs to go away. Jesus talked
            about shepherds because there was one over there in a
            pasture he could point to. But to bring in that imagery
            today and say, ``Pastor, you’re the shepherd of the flock,''
            no. I've never seen a flock. I've never spent five minutes
            with a shepherd. It was culturally relevant in the time of
            Jesus, but it's not culturally relevant any more.'' 

    \end{quote}

    He goes on to say, ``One of the criticisms I get is ``Your church is
    so corporate.''\ldots Bloggers complain, ``The pastor's like a
    CEO.'' And I say, ``OK, you’re right.  Now, why is that a bad
    model?''''

    So that's one major area in the fight for Biblical church
    leadership. The other is an area called ``egalitarianism''. There is
    a movement that has emerged out of liberal feminism called
    egalitarianism, and that movement asserts that women can and should
    be affirmed into pastoral leadership over churches just as much as
    men should. That error represents a rejection of the Bible's
    teaching on the role of women as the primary caretakers in the home,
    wives having their primary ministry in the home, and as a result
    there has been a shocking rise of just abysmal female ``teachers''
    like Rachel Held-Evans, Jory Micah, Sarah Young, Priscilla Shrier,
    and the Queen Maven of Unsound Doctrine, Beth Moore. To all those I
    say along with John MacArthur, ``Go home.''

    But those are the two major areas at stake in today's church when it
    comes to church leadership, and so we want to look tonight at what
    the Bible says about leaders in churches.

    Leadership is a tricky thing to understand. It seems natural that
    there is a ``two sides of the coin'' effect with the ideas of
    leaders and followers. On one side of the coin, you have leaders; on
    the other side, you have followers. That makes sense, doesn't it?
    But there is a thought that has developed today that because of that
    partnership between leaders and followers, then there is equal
    weight and emphasis on both leaders and followers. For example, I
    know of one pastor that I highly recommend who, whenever anyone
    brings up issues with the state of leadership in the Church today,
    he will always turn around and point to the followers as the root of
    the problem.

    I don't think that's quite accurate. I think the Bible puts a
    premium on leadership as having the responsibility to lead a local
    church, and whenever you see a church that goes astray, yes, the
    followers bear some responsibility for that, but I think that
    ultimately the burden and root of the issues lies with leadership.
    Let me give you just a couple texts that I think illustrate this.
    
    First, in Hebrews 13:17, Christians are commanded to ``Obey your
    leaders and submit to them---for they keep watch over your souls as
    those who will give an account\ldots'' Pastors and leaders in the
    church will give an account not just of themselves, but also of the
    flock they pastor. Think of it this way: when a pastor stands before
    the Lord, imagine the Lord saying, ``I gave you a charge to keep
    watch over Bob's soul in your church, and yet Bob persisted in a
    life of sin. Did you do anything to address it? Did you follow my
    command to preach the Word? Did you warn him? Did you even know of
    his life of sin?'' And there are many pastors today who will be able
    to do nothing but hang their heads in shame, because they did not
    keep watch over the souls of their flock, and so they will fail that
    account when they stand before the Lord.

    Another text that shows an emphasis on leadership is found in
    Ephesians 4:11--12. Paul writes, ``And He Himself gave some as
    apostles, and some as prophets, and some as evangelists, and some as
    pastors and teachers, for the equipping of the saints for the work
    of service, to the building up of the body of Christ\ldots'' Here we
    see that saints are to be equipped for ``the work of service'' which
    is defined as ''the building up of the body of Christ'', but notice
    how this equipping is accomplished. It is accomplished through the
    work of, among others, ``pastors and teachers''. And notice that
    pastors and teachers are singled out as being gifts that have been
    given by the Lord Jesus Christ Himself.

    So the Bible does in fact put a premium on leadership within the
    church, and the first thing to note for any particular church is the
    state of their leadership. So what are we to make of leadership
    within the church? Is there any Biblical guidance available to us
    for how to judge church leadership? Does the Bible say anything to
    us about what church leadership looks like, who should be a church
    leader, and what do church leaders do? 

    And of course, the answer is yes. The Bible says a great deal about
    leadership, and so tonight we're going to examine that. \\

    \noindent \textbf{Biblical Eldership} \\

    One of the best, most comprehensive volumes on the subject of church
    leadership is Alexander Strauch's \emph{Biblical Eldership}. It's
    the best resource I can recommend today, and it's not even close. In
    it, Strauch exposits every single passage in the Bible that touches
    on leadership in the church, and one of the main topics he addresses
    is that of elder qualifications. Who is qualified to be a pastor in
    a church? Let's look at those briefly before we get into our passage
    for tonight: \\

    \noindent \emph{Spirit-Given Motivation for the Task} \\

    The first qualification we find in Scripture is what is often termed
    a ``pastoral calling''. We often say that a man who wishes to be a
    pastor must be ``called'' by God. I think Strauch articulates it
    better by pointing to 1 Timothy 3:1 and identifying this call as a
    ``spirit-given motivation for the task''. In other words, God
    doesn't call pastors the same way as He did Old Testament prophets,
    for example. Rather, God gives men a desire and a burden to
    accomplish the task of elder, and part of working at the task
    includes being recognized in the office of pastor. We don't have
    time to exposit through 1 Timothy 3:1 completely, but I can safely
    say that what Paul has in view here when he writes ``If a man
    desires the office of elder, he desires a noble thing,'' is that the
    desire is for the task, not the status. And so the first
    qualification is that a man who would lead a church must have a
    burden for serving the Lord in that way. \\

    \noindent \emph{Exemplary Character Qualities} \\

    The next set of qualifications we find in both 1 Timothy 3 and the
    parallel passage in Titus 1 is that leaders in the church must
    possess exemplary character qualities. That is, they must be of
    impeccable moral character such that they reflect God's holiness and
    do not bring dishonor upon the Lord or the church. And just looking
    at these lists for a moment, we see that the first, over-arching
    qualification is to be ``above reproach''. The idea of being above
    reproach functions in two ways: it acts as a general catch-all to
    describe someone of impeccable character, and it also acts as an
    umbrella term to capture all the rest of the character qualities
    that follow in these passages. The idea of above reproach means that
    there's not always something going on with a leader, there's not
    any unresolved issues, and that any issue must be fully examined and
    understood before rendering a leader unfit for office.

    Following that over-arching qualification, Paul gives us a familiar
    list of moral imperatives that should describe the manner of life of
    a church leader. I want to go over these quickly and say just a few
    remarks about them before we get to our main passage tonight:

    \textbf{Husband of one wife} In both 1 Timothy and Titus, this
    is the first item following the requirement to be above
    reproach.  Thus, this is the foremost way in which a man can
    demonstrate himself to be above reproach.  He is to be a
    faithful man who is committed to sexual purity, and if married
    is committed to being faithful to his wife.  Literally
    translated, this qualification means that an elder must be ``a
    one woman man.''

    \textbf{Sober-minded, disciplined} Also known as ``temperate'', this
    qualification requires self-control and measured judgement; it
    means an elder must not be given to rash or excessive behaviour.
    An elder must be restrained and clear-headed at all times, able to
    exercise sound judgment in the matters he encounters throughout
    the course of his life.

    \textbf{Self-controlled} Also known as ``prudent'', this is
    related to previous qualification of being \emph{sober-minded}; it
    focuses specifically on the discretion needed to deal with people
    and their problems.

    \textbf{Respectable} This qualification indicates that an elder
    should be well-behaved and orderly in the way in which they live
    their lives.  Alexander Strauch indicates that ``although the word
    is used to describe properness in outward demeanor and dress in 1
    Timothy 2:9, its usage here conveys the more general meaning of
    `orderly...well-behaved, or virtuous...that which causes a person
    to be regarded as respectable by others.''

    \textbf{Hospitable} While this is a simple concept, it is also an
    often overlooked qualification for pastors.  Citing several
    Biblical commands to practice hospitality, Alexander Strauch
    writes, ``These New Testament commands to practice hospitality are
    all found within the larger context of Christian love...for an
    elder to be inhospitable is a poor example of Christian love and
    care for others.''

    \textbf{Not a Drunkard} This is an obvious one, but Alexander
    Strauch writes regarding drunkenness:
    
    \begin{quote}

        An elder must be above reproach in his use of alcohol.  Paul
        uses strong language here that means not preoccupied or
        overindulgent in wine.  Drunkenness is sin, and persistently
        drunken people required church discipline...So a person in a
        position of trust and authority over other people can't have a
        drinking problem...If an elder has a drinking problem, he will
        lead people astray and bring reproach upon the church.

    \end{quote}

    \textbf{Not Violent but gentle, not arrogant} These
    qualifications stand in contrast to each other.  In fact, out of
    all the qualifications, gentleness is the one qualification that
    best describes the manner in which God treats His people as
    sinners.  It is actually a word that carries with it the idea of
    God's unmerited favor in saving us from our sins when all we
    deserved was an eternal punishment in Hell.  So rather than
    being quick to exact retribution by way of violence, and elder
    must demonstrate the same grace to others that God has
    demonstrated toward him.

    \textbf{Not quarrelsome, not quick tempered} The idea here is
    someone who always has to be right, rather than letting an
    insignificant matter drop.  There is a time to stand your ground
    and be right; for example, when dealing with the Word of God, we
    want to take great pains to get that right, and we want to
    defend it when people malign and distort it.  But that is a far
    cry from the prideful ambition to assert one's self over others
    in some trivial situation, such as the outcome of a sporting
    event or the weather forecast.  Thus, an elder must demonstrate
    a life of uncontentiousness.

    \textbf{Not a lover of money} Is the driving force behind a man's
    desire for the pastorate simply a paycheck?  If so, then that man
    is disqualified according to this passage.  This is more than just
    a stated goal or a spoken word; it has to do with how a man is
    living his life.  Is he always seeking for personal profit, even
    while maintaining at least verbally to the contrary?  Then that is
    a disqualifying blight of character.

    \textbf{A lover of good, holy, upright} All of these
    qualifications demonstrate the idea of above reproach, and can
    be summed up with the term \emph{holy}.  A qualified elder is a
    man who loves what is good and hates what is evil; he is
    upright, and he is holy. Alexander Strauch writes of this
    qualification,
    
    \begin{quote}

        To be ``devout'' is to be firmly committed to God and His Word.
        It is to be separated unto God and pleasing to God.  Despite the
        changing winds of culture and circumstances, the devout person
        faithfully clings to God and His Word.

    \end{quote}

    Now, I would love to take several weeks and exposit through each of
    these, but what I want to highlight two things about these
    lists. First, both of these lists of pastoral qualifications are
    written in the present-tense. Meaning, they are qualities that a man
    must possess \emph{now}. A man at one point in his past life may
    not have met these qualifications, and that doesn't matter. The
    issue is how is he living his life now? Second, don't think for a
    moment that just because this is a list for qualifications of pastor
    that this list doesn't apply to you. We're going to see that in a
    moment, but suffice to say that the reason these character
    attributes are listed is so that the man of God will be qualified
    for leadership, and then we, as those who submit to leadership, will
    be able to watch those qualities lived out in his life and then
    develop and emulate them in us! In other words, if you want to know
    what being self-controlled looks like, what being peaceable looks,
    what being righteous looks like, examine the lives of your leaders
    and imitate them in those things. \\

    \noindent \emph{Made Skillful for the Job} \\

    The final category of qualification that Strauch makes is to
    highlight that a pastor is to be gifted for the job. Specifically,
    the pastor is to have the ability to teach sound doctrine in
    accordance with God's Word and explain and apply the Scriptures to
    the flock so that they will know and understand what God's
    prescriptions, precepts, principles, and prohibitions are and how to
    live those out in our lives. Both 1 Timothy 3 and Titus 1 include a
    requirement to ``be able to teach'', and Titus 1 further explains
    that one of the reasons for this is to be able to refute false
    teaching and ``reprove those who contradict'' it. Not everyone is a
    teacher, and those that are teachers and leaders in a church are
    held to a higher standard (James 3:1), and so Paul lays that out as
    a requirement for leadership in the church. \\

    \noindent \textbf{1 Peter 5:1--3} \\

    So, all of that is introduction, and we could weeks just poring
    over all of those things, but I want now to take a step back and
    look at the bigger picture. What is going on with all of these
    things, how do they fit into the larger design that God has for
    building His Church? And the key question is this: What is it that
    elders are supposed to do? What should we expect elders to do in the
    performance of their job? That's really where the rubber meets the
    road, because this is where right leadership is distinguished from
    bad leadership.

    Alexander Strauch points out several sweeping phrases that describe
    what elders are expected to do as the leaders of the church. Using
    Biblical descriptions, He points out that elders are to feed the
    flock. Elders are to preach the Word. Elders are to guard the flock.
    Elders are to lead the flock.  We might expect elders to conduct
    some sort of visitation or ministry to those who are chronically ill
    in the congregation. All these things are things we might consider
    as part of the job description of pastor.

    But there is one passage that I find in Scripture that really sums
    up the totality of pastoral ministry. All other functions or
    ministry imperatives we find in Scripture can really be fit under
    this one particular command we find in this passage. And that is
    Peter's command in 1 Peter 5:1--4 to ``shepherd the flock.'' So turn
    with me there now and we're going to spend the rest of the time
    expositing our way through this passage to see the essence of what
    Right Leadership will look like in a church.

    Let me read through 1 Pet. 5:1-4. Peter writes:

    \begin{quote}

        ``Therefore, I exhort the elders among you, as your fellow elder
        and witness of the sufferings of Christ, and a partaker also of
        the glory that is to be revealed, shepherd the flock of God
        among you, exercising oversight not under compulsion, but
        voluntarily, according to the will of God; and not for sordid
        gain, but with eagerness; nor yet as lording it over those
        allotted to your charge, but proving to be examples to the
        flock.  And when the Chief Shepherd appears, you will receive
        the unfading crown of glory.''

    \end{quote}

    We see first in this passage that this is an apostolic exhortation.
    But first notice that Peter identifies himself as ``your fellow
    elder''. Peter was an apostle. In fact, he wasn't just a
    run-of-the-mill journeyman apostle. Peter was one of the foremost
    apostles. Paul describes Peter in Galatians 2 as being one of three
    apostles who were ``pillars'' in the early church. Peter was among
    the first apostles to spread the gospel to non-Jews, recalling the
    story of Cornelius.  And yet, here is Peter, lowering himself from
    the status of apostle, and exhorting his readers as a ``fellow
    elder''. So from the get-go, Peter displays incredible humility.

    He then shows that same humility is to be expected of all elders.
    What is his exhortation? ``Shepherd the flock of God \emph{among}
    you.'' It's not the flock of God \emph{under} you. It's not the
    flock of God which you command. Shepherds are to be \emph{among}
    their flock. So, here is our first lesson on church leadership:
    Peter has lowered himself from an apostle to a fellow elder, and
    then he has taken all the other elders along with himself, and
    lowered everyone to be on the same level as the rest of the flock.

    I can recall the exact moment at which I made up my mind to become a
    member here. Ed was teaching through a series on Sunday nights, and
    one night he said, ``I am just a sheep along with the rest of you.''
    Immediately I thought of this passage, and that was the
    first time I've ever heard a pastor publicly say that. I was already
    satisfied with the church's doctrine, and I wasn't aware of any
    shenanigans going on inside the church, but when Ed said that, I
    thought, ``He gets it, and that's someone I should submit to.'' And
    so I'm thankful for Ed and Brett as elders who get this.

    Now, let's look at this exhortation: Peter writes, ``shepherd the
    flock of God, exercising oversight\ldots'' This is the umbrella term
    that describes everything a pastor is to do in his service to the
    church. Everything a pastor does, whether that's preaching, whether
    that's visitation, whether that's praying and counselling, whether
    that's enacting church discipline, whether that's correcting error
    or protecting the church from false teaching, or whether that's
    administrating the church, all of those things are part of the
    shepherding tasks that fall under Peter's exhortation here. Peter
    uses two terms here: First, he uses the word \emph{poimaino}, which
    is the verb form of the word ``shepherd''.  And then he says that
    elders are to shepherd the flock by ``exercising oversight''. That
    word that Peter uses is \emph{episkopeo}; it's where we get our word
    ``episcopal'' or ``bishop''. It literally means ``look after
    carefully''. What is a shepherd? Someone who herds sheep.  What does
    a shepherd do? Looks after the sheep carefully. What does that mean?

    It means everything. Throughout the Old Testament, Yahweh, the God
    of Abraham, Isaac, and Jacob was known as The Shepherd of Israel.
    Psalm 23, the most well-known psalm of all time begins, ``The
    L\textsc{ord} is my \emph{shepherd}.'' Jesus, in John 10, showed
    Himself to be the Great Shepherd. Shepherds were everything to their
    sheep, and sheep were everything to their shepherds. 

    Now, you might think of modern sheep-herding in terms of sheep-dogs
    nipping around the sheep to corral them into place. Or maybe you
    think of sheep as just docile creatures who you put into the fenced
    pasture and let them graze all day. But that's not what Peter meant.
    In Peter's time, a shepherd was everything to the sheep. Sheep are
    dumb, essentially helpless animals. They need looking after. There
    weren't any fenced pastures that the shepherd just dumps the sheep
    off into and then goes away. The shepherd was constantly there with
    his sheep. If a wild animal tried to make off with one of the sheep,
    the shepherd killed the intruder or drove it away. If one of the
    sheep was hurt, maybe a broken leg or something, the shepherd would
    bind the wound and then carry the sheep on his shoulders until the
    sheep could walk again. The shepherd would walk in front of his
    sheep, and the sheep would follow the sound of his voice. They were
    actually voice-imprinted on the individual shepherd. The shepherd
    would walk and find good, safe, plentiful grassy areas where the
    sheep could be fed and watered. The shepherd would lead the sheep
    back to the community sheep pen and stand in the gate. He would
    lower his rod to block the sheep from rushing into the sheep pen all
    at once, and instead they would pass under his rod one-by-one so he
    could examine each sheep for wounds, to make sure they were
    nourished, etc. Sheep have really oily wool, it's called lanolin,
    and that wool gets matted in their hindquarters when the sheep use
    the bathroom. It gets caked up in there and the sheep get blocked
    up, so the shepherd would reach up into the sheep's hindquarters and
    clean out the sheep's mess from when it used the bathroom. All of
    those things are involved in shepherding.

    Now consider this: if you're here tonight and you're a Christian,
    you're a sheep. And when Peter exhorts elders to shepherd the flock
    of God, that implies that you need to be shepherded. It means that
    if you're not being shepherded, then you're a sheep without a
    shepherd.  And the way that you fix that is to place yourself in the
    care of a team of pastors who will be responsible for feeding you
    from God's Word, for protecting you from the ravenous wolves like
    the Andy Stanley's and Beth Moore's that are out there, who will
    bind up your wounds and bring you spiritual healing and counsel, and
    who will whack you on the head with the rod of God's Word and bring
    you back when you need whacking.

    That's the idea that Peter has in ``shepherding'' here. But notice
    that Peter gives us three descriptions as to how elders are to look
    after the flock carefully as they shepherd: it is to be done
    willingly, with eagerness, and as an example. And so let's look at
    each of those. \\

    \noindent \emph{Willingly} \\

    Notice that Peter provides us with some clarification: he writes
    that elders should be ``not under compulsion''.  The word here is
    derived from the Greek word \emph{anagke}, which literally means ``a
    bent/uplifted arm poised to meet a pressing need.''  The idea is a
    heavy weight that is coming down on top of you, and you have your
    arms lifted up to keep that weight from bearing down on you and
    crushing you, and you are slowly but inevitably giving in to the
    pressure. That is not how elders are to be performing their job. \\
    
    Instead, Peter says that an elder is to shepherd the flock
    \emph{willingly}.  The word used here is \emph{hekousios}, which
    literally means ``of one's own accord, spontaneously''.  This
    indicates that elders should be shepherding out of the overflow of
    their heart. Elders are to have a spirit-given motiviation and
    desire for serving the Lord and serving others in this way. \\

    \noindent \emph{Eagerness} \\

    The next way that Peter stipulates as to the role of a shepherd is
    that they are to serve with \emph{eagerness}.  The word here is
    \emph{prothumos}, which means ``passion shown in advance, i.e.,
    pre-inclined, ``thoroughly willing''''.  It's interesting that Peter
    contrasts this with ``shameful gain''.  When Peter cites that trait,
    he is referring to a greedy spirit that is eager for base gain.  In
    other words, if an elder is shown to be eager for his office only
    after the possibility of a paycheck presents itself, then that is an
    abuse.  An elder should be demonstrating a willing and cheerful
    eagerness to shepherd the flock regardless of the possibility of
    receiving any type of compensation or recognition. Again, this hits
    at where a pastor's heart is supposed to be. Rather than being
    focused on what he can get out of it, a pastor is to have the same
    mind as Christ, ``Who, being in very nature God, did not consider
    equality with God something to be grasped, but made himself nothing,
    taking the very nature of a servant, being made in human likeness.
    And being found in appearance as a man, he humbled himself and
    became obedient to death-- even death on a cross!'' Philippians 2.
    \\

    \noindent \emph{Examples} \\

    This leads us to the final way that Peter exhorts shepherds to
    oversee the flock. He first says that shepherds should not be
    \emph{domineering}.  The word, \emph{katakurieuo}, literally means
    to ``exercise decisive control as an owner with full jurisdiction.''
    The idea is an overbearing lord who subjugates and domineers his
    people.  This specifically prohibits the ``CEO'' model of the
    pastorate that we see plaguing so many churches today.

    Also notice the way in which Peter relates the shepherd to the
    flock: he describes the flock as being ``those in your charge''.
    The idea of ''those in your charge'' is best expressed in the King
    James rendering.  It reads, ``Neither as being lords over
    \emph{God's heritage}''.  In other words, a shepherd isn't to view
    the flock as something to be in charge of, but the flock that elders
    are supposed to shepherd is to be viewed as God's own inheritance
    that He purchased with His own blood by His own sovereign grace!

    Rather than those negative prohibitions, pastors are to be
    \emph{examples}. The word is \emph{tupos}, which means ``a model
    forged by repetition''.  It comes from the idea of a wax stamp
    struck by a die.  Figuratively, the idea is to be ``a reliable
    precedent for others to then follow''. Now, what example do you
    think Peter is referring to here?  It's no less than the example of
    Christ. Pastors are to model Christ in the course of their
    shepherding. 

    How is that to happen? Well, I just gave you literal definition of
    the term: ``a model forged by repetition.'' What repetition is to
    occur? Is there anything that pastor might do, over and over and
    over again, week after week, that might possibly result in looking
    more like Christ? I believe this can only be a reference to pastors
    immersing themselves repetitively in Scripture. Week after week,
    pastors should be committing themselves to the study and exposition
    of God's Word. And in doing so, God's Word is to infuse their lives
    such that by the time the pastor brings the message to us on Sunday
    mornings, he will have already wrestled with that text in his own
    life and as a result will have been further conformed to Christ so
    that he can say, ``This is what God's Word says, and oh by the way,
    here is what the looks like.''

    This is why the primary task of shepherd is found in Paul's command
    to Timothy to ``preach the Word.'' The primary task a pastor does to
    preach, proclaim, and exposit God's Word before the people. It's not
    the only task, but it is the primary task. If we all of a sudden
    noticed that Ed was spending most of his time out visiting with us,
    or keeping his bees, or administering the church finances, and he
    was doing those things at the expense of his study of the
    Scriptures, we would have a problem. And I'm thankful that he does
    spend most of his ``pastor'' time in studying and preparing for each
    message. That's the example we should take note of.

    So we see that rather than today's CEO model of church leadership,
    elders are to shepherd the local church in such a way as to model
    the very life of Christ before us, so that we too may know how to
    grow up into His image. \\

    \noindent \textbf{Conclusion} \\

    Well, as we close let's take note of the encouragement that Peter
    gives to those would undertake this task of shepherding the flock.
    He says, ``when the chief Shepherd appears.'' This ties together the
    servant attitude we've been hinting at. Elders aren't the ones in
    charge of the church; they're servants, or what we call
    ``under-shepherds'', and they operate under the orders of the Chief
    Shepherd. And He's going to appear one day, and one of the things
    He's going to do is render judgment on those who undertook this
    task. And for those that acquit themselves well, who did in fact
    shepherd the flock by looking carefully over them with a willing
    heart and with eager dedication as examples, the Chief Shepherd will
    hand them a crown of glory that will not fade.  Think of that.
    Shepherds are to deny themselves the pursuit of worldly gain. And
    what they get in return is so much more than what they might have
    lost: they get eternal reward!

    So in this crisis of leadership we see in churches today, we need
    church leaders who fulfill these characteristics that Peter exhorts
    us towards. Leadership is critical, and in considering any
    particular church, you should take a consideration for how its
    leaders measure up to the Scriptural standards for shepherds of the
    flock. Let's pray:

    \begin{quote}

        Dear Father, we thank you for being the Shepherd of Your people,
        and we thank you for giving us the gift of leaders who shepherd
        us through the proclamation of Your Word. Please purify your
        Church of the hirelings who plague the flock today through
        shameful gain and sordid domination, and cause those who are in
        such places to come out from them and find green pastures in
        churches such as this one. We ask this in Your Son's Name and
        for His Glory, Amen.

    \end{quote}

\end{document}

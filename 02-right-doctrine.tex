\documentclass[letterpaper, 12pt]{article}

\usepackage[margin=1in]{geometry}

\begin{document}

    \thispagestyle{empty}

    \begin{center}

        \textbf{\large{What To Look For In A Church: Right Doctrine}}

    \end{center}

    \noindent \textbf{Introduction} \\

    Good evening. Once again we continue our study on What To Look For
    In A Church, and tonight we're going to look at the second part:
    Right Doctrine. Now, last week we saw that the first marker of a
    healthy church is Right Leadership, where a church is led by godly,
    gifted men who are committed to shepherding the flock as examples to
    the rest of the congregation. Tonight, we're going to cover the
    necessity of Right Doctrine, and next week, Lord willing, we'll look
    at the Right Practice of a church, or what a church does and should
    look like in its corporate life.

    You can turn with me to 1 Timothy 4, but as we study our passage
    tonight I want to say a few remarks at the outset about doctrine in
    general. First, the word ``doctrine'' is the English-ization of the
    Greek word \emph{didasko}. It means, simply, ``teaching''. So when
    we talk about Bible doctrine, all we mean is simply what does the
    Bible teach about a particular subject.

    For example, we can talk about The Doctrine of Justification By
    Faith, which refers to the Bible's teaching about how a sinner is
    made righteous before a Holy God. We can talk about the Doctrine of
    Total Depravity, which means the Bible's teaching about man's sin
    nature and its effects. You've heard of the Doctrine of Election,
    which means the Bible's teaching on God's sovereign choice to save a
    people for Himself.

    So the idea of ``doctrine'' isn't really some elite, academic
    concept that you need a PhD to understand. Whenever you hear someone
    talking about ``doctrine'', just know that all they mean is ''what
    the Bible teaches''. Now, some doctrines are more difficult to
    understand that particular doctrine; Peter says as much as the end
    of 2 Peter, where he refers to Paul's letters as Scripture and says
    that some of the things Paul writes are hard to understand. But the
    general concept of what doctrine is isn't hard to grasp: it simply
    means ''teaching''.

    Now, throughout church history there have been various battles the
    church has fought when a particular doctrine has come under assault.
    In the early church, the Doctrine of the Trinity was threatened by
    Arius. Then, following that, the Doctrine of God's Sovereignty was
    maligned by Pelagius. The Reformation revolved around the Doctrines
    of Sola Scriptura and Sola Fide: Scripture Alone and Faith Alone,
    the Doctrine of Justification by Faith. So there has been really a
    constant assault on various fronts throughout the ebb and flow of
    history as Christ has built His Church.

    But as far as I can tell, there has never been a time throughout the
    last 2,000 years like today. Today, doctrine is under assault like
    never before. On virtually every front, there is some naysayer that
    is either trying to tear down long-standing, apostolic truth, or
    some whacko that is claiming to have received new revelation that
    supercedes millenia of established, clear Bible doctrine. And it's
    not just happening from the outside world, it's also happening
    within the church.

    Now, I could spend our time talking about the various doctrines that
    are under threat today. I could point to the the movement in
    Reformed circles that is redefining the Doctrine of Total Depravity
    and seeking to limit in an unbiblical way to merely the Doctrine of
    Total Inability. That's a threat. I could point to the total failure
    of churches in the Doctrine of Worship, where churches have turned
    worship into nothing more than the pursuit of an emotional high
    instead of recognizing that we are to worship God with fear and awe
    because He is in fact a consuming fire, not filled with ``glory and
    freedom'', but with the precedent of reducing two of His foremost
    worship leaders to piles of ash because they offered ``strange
    fire''. I could highlight the increasing denial of the Doctrine of
    Inerrancy, whether that's by those such as Tim Keller and the
    Biologos organization that would attempt to redefine Genesis 1 and 2
    to make it compatible with the godless and disproven theory of
    evolution, or by those who would seek to reverse the God-design
    roles for men and women, making men effeminate, man-bun wearing
    non-leaders and women masculine, offputting usurpers who both
    abandon their God-ordained roles in society.

    But along with those serious errors, there is another lie
    underpinning them all. There is a common miscategorization of
    doctrine today as being somehow unimportant. We have created these
    categories of ``primary'' doctrines, ``secondary'' doctrines, and
    even ``tertiary'' doctrines, and those are seemingly arbitrary. The
    Bible nowhere makes such categorizations, and while it does say in 1
    Corinthians 15 that the the Gospel is of ``first importance'', that
    does not imply that other so-called ``non-Gospel'' doctrines are
    therefore unimportant. Paul talks about in Romans 14 ``matters of
    opinion'', and he is not referring there to Biblical teaching about
    doctrine. What the Bible teaches is not a matter of opinion, and
    it's not unimportant. The infinite, all-powerful, creator God of the
    universe, took the time to painstakingly author a miraculous, divine
    book of His Thoughts for what He thought we need to know in order
    for our eternal souls to be saved, and He didn't put fluff in there.

    Our passage this evening is going to help disabuse you of that
    notion, and so let's look at 1 Tim. 4:6--16. I'm going to show you
    how Paul not only makes doctrine paramount, but he also commands it.
    So not only is doctrine a matter of extreme importance, it's also a
    matter of obedience. I want to focus specifically on verse 16, but
    for context I'll start reading in verse 6. The Apostle Paul writes,

    \begin{quote}

        In pointing out these things to the brethren, you will be a good
        servant of Christ Jesus, constantly nourished on the words of
        the faith and of the sound doctrine which you have been
        following. But have nothing to do with worldly fables fit only
        for old women. On the other hand, discipline yourself for the
        purpose of godliness; for bodily discipline is only of little
        profit, but godliness is profitable for all things, since it
        holds promise for the present life and also for the life to
        come. It is a trustworthy statement deserving full acceptance.
        For it is for this we labor and strive, because we have fixed
        our hope on the living God, who is the Savior of all men,
        especially of believers.

        Prescribe and teach these things. Let no one look down on your
        youthfulness, but rather in speech, conduct, love, faith and
        purity, show yourself an example of those who believe. Until I
        come, give attention to the public reading of Scripture, to
        exhortation and teaching. Do not neglect the spiritual gift
        within you, which was bestowed on you through prophetic
        utterance with the laying on of hands by the presbytery. Take
        pains with these things; be absorbed in them, so that your
        progress will be evident to all. \textbf{Pay close attention to
        yourself and to your teaching; persevere in these things, for as
        you do this you will ensure salvation both for yourself and for
        those who hear you.}

    \end{quote}

    So, before we delve into verse 16 more deeply, I want to just look
    with you at an overview of this entire passage, and I want to point
    out some observations by way of context, and then we'll look at
    Paul's flow of thought leading into verse 16, and then we'll parse
    out verse 16 itself and see what kind of conclusions we can draw
    from it. Sound good? Okay, here we go.

    First, I want to highlight that this entire passage is all about
    doctrine. Paul is focused here on doctrine. And to show you that, I
    want to point out the number of times Paul uses the word, or a root
    of it, for ``doctrine''. First, in verse 6. Paul says, ``In point
    out these things to the brethren, you will be a good servant of
    Christ Jesus, constantly nourished on the words of the faith and of
    the sound \emph{doctrine} which you have been following.'' There it
    is, \emph{didaskalia}, or ``doctrine'', as you see there.

    Now skip down to verse 11. We see that Paul writes, ``Command and
    \emph{teach} these things.'' The word ``teach'' there is the verb
    form of the word ``doctrine''. 

    Now look in verse 13. Paul says to ``devote yourself to the public
    reading of Scripture, to exhortation, to \emph{teaching}.'' There's
    the word for ``doctrine'' again. We could have just as easily
    translated this, ``devote yourself to the public reading of
    Scripture, to exhortation, to \emph{doctrine}.''

    And then finally, in our verse tonight, verse 16, we read, ``Keep a
    close watch on yourself and on the \emph{teaching}.'' And again,
    that's the word \emph{didaskalia}, or ``doctrine''. 

    So, four times in this section Paul refers to ''doctrine''. He
    starts this section to Timothy by emphasizing doctrine, he
    continually pushes Timothy to focus on doctrine, even commanding it,
    and then he closes this section by emphasizing doctrine. There's no
    question that in Paul's mind, doctrine was paramount.

    The second thing I want to look at is the words that Paul uses to
    emphasize doctrine. They are strong words. Words of command. In
    verse 6, Timothy is to be nourished in doctrine. In verse 13,
    Timothy is to be devoted to doctrine. In verse 15, Timothy is to be
    immersed in doctrine. And in verse 16, Timothy is ``keep a close
    watch'' on doctrine and persist in it.  Let's look at each of those
    words a little more fully.

    In verse 6, Paul commands Timothy to be \emph{nourished} or
    \emph{trained} in doctrine.  The word Paul uses is \emph{entrepho}.
    It means literally ``to be nourished''. The root words means
    ``feed''. Now, what's the purpose of food? It's to sustain life.
    When we eat physical food, we are taking in the nutrients and the
    vitamins and the energy our bodies need to continue function as God
    designed them, so that we may use our bodies to accomplish that
    which God wills. But where does our spiritual life come from? How do
    we take in the nutrients into our spiritual life so that we are
    equipped to be spiritually healthy followers of Christ? That happens
    by being taught doctrine. In other words, if you're not ingesting
    sound, Biblical doctrine on a regular basis, you are a malnourished
    and spiritually anemic individual.

    In verse 13, Paul writes that Timothy is to ``give attention to''
    doctrine. The Greek word is \emph{prosecho}, meaning ``to give full
    attention.'' It's a compound word from two words meaning ``to hold
    your mind toward.'' The idea is where do you place your focus? If
    you were a visitor to the church at Ephesus where Timothy was
    ministering, Paul wanted you to come away from that visit saying,
    ``Man, that church is focused on doctrine.''

    Now look at verse 15, we see Paul commanding Timothy to ``Take pains
    with these things; be absorbed in them.'' Paul here is referring to
    the commands toward doctrine that he just wrote to Timothy. And he
    says Timothy is to ``take pains'' with them. Literally, it's to
    ``revolve the mind'' around them, or to ''exercise myself in''.
    Other synonyms are ``study'', ``ponder'', or ``premeditate''. And
    then Paul says Timothy is to ``be absorbed in them''. This is
    literally translated, ``exist in them''. Timothy's waking thought,
    his whole life, is to be consumed with these things. His entire bent
    of life is to be driven by the teaching found in Scripture, and he
    is to take great pains to exercise his mind towards learning them.

    And finally, we come to our verse tonight, verse 16. Paul writes to
    Timothy, ``Pay close attention to yourself and to your teaching,''
    or ``doctrine'', ``persevere in these things, for as you do this you
    will ensure salvation both for yourself and for those who hear you.''

    Now, I want to walk you through the flow of Paul's thought starting
    in verse 6 to lead you up to this verse. Paul starts this section by
    telling Timothy that as Timothy teaches the church the doctrine on which he has
    been nourished, that he will be a good servant of Christ. He
    warns Timothy against becoming enamored with the lies of the world,
    but instead Timothy is to be trained in this sound doctrine so that
    he will be more and more Godly. This is important and trustworthy
    because spiritual training is far more profitable than mere earthly,
    physical training. 

    But notice the motivation Paul gives for doing so in verse 10: Paul
    says, ``For it is for this [the promise for the present life and
    also for the life to come] we labor and strive, \emph{because we
    have fixed our hope on the living God}.'' In other words, Paul is
    connecting training in sound doctrine as an outcome of salvation.
    The one goes with the other. You become a believer; you then ``grow
    in the grace and knowledge of our Lord and Savior Jesus Christ.''
    The motivation for the Christian, then, should be, ``Because God saw
    fit to rescue me, a wretched sinner, by purchasing me with the blood
    of His own Son, then I want to learn everything the Bible teaches
    about God, who He is, how He works, what He does, and what He wants
    me to be like!''

    And then in verse 11, we see that the reason Paul is commanding
    Timothy in these things is so that Timothy might be an example to
    the church at Ephesus. Doesn't that sound familiar from last week?
    Timothy is to be, get this, ``a model forged by repetition'' to the
    other believers at Ephesus. Remember what is to be repeated? Day
    after day, week after week, Timothy is to be devoted to training
    himself in sound doctrine so that his life will reflect Christ's
    teachings to the rest of the church. 

    And so because of that, Paul then presses on Timothy the importance
    and the emphasis on teaching doctrine. There's an insistency there;
    Paul is almost urgent. He's saying to Timothy over and over:
    ``Timothy, doctrine is your life. Devote yourself to the scriptures
    and the doctrine they teach. Take pains with them. Do not neglect
    the gifting you have been given in this area. It is for a purpose,
    therefore, absorb yourself in it. Give yourself wholly to it!''

    And all of that pressing, all of that hammering home that Paul is
    doing to poor young Timothy comes to this rising crescendo in verse
    16, where Paul culminates by telling Timothy, ``Pay close attention
    to yourself and to your teaching; persevere in these things, for as
    you do this you will ensure salvation both for yourself and for
    those who hear you.'' So let's dive into this verse and see what
    Paul is telling Timothy.

    Paul tells Timothy to ``pay close attention''. This is an
    interesting word. Earlier in verse 13, Paul told Timothy to ``give
    attention to''. This is like that, only moreso. Stronger. More
    intense. It's not just, ``Timothy, pay attention to'', it's
    ``Timothy, pay \emph{close} attention to''. It means to pay
    attention to something, and then keep your attention on it. Focus on
    it to the exclusion of all other distractions.

    Let me show you an example from Scripture of what that looks like.
    Turn with me to Acts 3. At the beginning of this chapter, Peter and
    John heal a lame beggar. And look in verse 3: ``Seeing Peter and
    John about to go into the temple, he asked to receive alms. And
    Peter directed his gaze at him, as did John, and said ``Look at
    us.'' And he fixed his attention on them, expecting to receive
    something from them.''

    Do you see that part where the lame beggar fixed his attention on
    Peter and John? That's the same word. That beggar paid close
    attention to the apostles. He wasn't going to let his attention
    waver for one second, because he was going to get something from
    them.

    That's the same way Timothy is to treat these things that Paul
    writes about. He is expecting to gain Godliness from it, and
    therefore, Timothy should not let his attention waver from the
    pursuit and training of doctrine. ``Pay close attention.''

\end{document}

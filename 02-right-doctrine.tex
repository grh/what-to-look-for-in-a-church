\documentclass[letterpaper, 12pt]{article}

\usepackage[margin=1in]{geometry}

\begin{document}

    \thispagestyle{empty}

    \begin{center}

        \textbf{\large{What To Look For In A Church: Right Doctrine}}

    \end{center}

    \noindent \textbf{Introduction} \\

    Good evening. Once again we continue our study on What To Look For
    In A Church, and tonight we're going to look at the second part:
    Right Doctrine. Now, last week we saw that the first marker of a
    healthy church is Right Leadership, where a church is led by godly,
    gifted men who are committed to shepherding the flock as examples to
    the rest of the congregation. Tonight, we're going to cover the
    necessity of Right Doctrine, and next week, Lord willing, we'll look
    at the Right Practice of a church, or what a church does and should
    look like in its corporate life.

    You can turn with me to 1 Timothy 4, but as we study our passage
    tonight I want to say a few remarks at the outset about doctrine in
    general. First, the word ``doctrine'' is the English-ization of the
    Greek word \emph{didasko}. It means, simply, ``teaching''. So when
    we talk about Bible doctrine, all we mean is simply what does the
    Bible teach about a particular subject.

    For example, we can talk about The Doctrine of Justification By
    Faith, which refers to the Bible's teaching about how a sinner is
    made righteous before a Holy God. We can talk about the Doctrine of
    Total Depravity, which means the Bible's teaching about man's sin
    nature and its effects. You've heard of the Doctrine of Election,
    which means the Bible's teaching on God's sovereign choice to save a
    people for Himself.

    So the idea of ``doctrine'' isn't really some elite, academic
    concept that you need a PhD to understand. Whenever you hear someone
    talking about ``doctrine'', just know that all they mean is ''what
    the Bible teaches''. Now, some doctrines are more difficult to
    understand that particular doctrine; Peter says as much as the end
    of 2 Peter, where he refers to Paul's letters as Scripture and says
    that some of the things Paul writes are hard to understand. But the
    general concept of what doctrine is isn't hard to grasp: it simply
    means ''teaching''.

    Second, there is a common miscategorization of doctrine today as
    being somehow unimportant. We have created these categories of
    ``primary'' doctrines, ``secondary'' doctrines, and even
    ``tertiary'' doctrines, and those are seemingly arbitrary. The Bible
    nowhere makes such categorizations, and while it does say in 1
    Corinthians 15 that the Doctrine of the Gospel is of ``first
    importance'', that does not imply that other so-called
    ``non-Gospel'' doctrines are therefore unimportant. Paul talks about
    in Romans 14 ``matters of opinion'', and he is not referring there
    to Biblical teaching about doctrine. What the Bible teaches is not a
    matter of opinion, and it's not unimportant. The infinite,
    all-powerful, creator God of the universe, took the time to
    painstakingly author a miraculous, divine book of His Thoughts for
    what He thought we need to know in order for our eternal souls to be
    saved, and He didn't put fluff in there. I'll speak more to this
    idea of doctrine being unimportant later, but I just wanted to get
    that out there now.

    Our passage this evening is going to help disabuse you of that
    notion, and so let's look at 1 Tim. 4:6--16. I want to focus
    specifically on verse 16, but for context I'll start reading in
    verse 6. The Apostle Paul writes,

    \begin{quote}

        In pointing out these things to the brethren, you will be a good
        servant of Christ Jesus, constantly nourished on the words of
        the faith and of the sound doctrine which you have been
        following. But have nothing to do with worldly fables fit only
        for old women. On the other hand, discipline yourself for the
        purpose of godliness; for bodily discipline is only of little
        profit, but godliness is profitable for all things, since it
        holds promise for the present life and also for the life to
        come. It is a trustworthy statement deserving full acceptance.
        For it is for this we labor and strive, because we have fixed
        our hope on the living God, who is the Savior of all men,
        especially of believers.

        Prescribe and teach these things. Let no one look down on your
        youthfulness, but rather in speech, conduct, love, faith and
        purity, show yourself an example of those who believe. Until I
        come, give attention to the public reading of Scripture, to
        exhortation and teaching. Do not neglect the spiritual gift
        within you, which was bestowed on you through prophetic
        utterance with the laying on of hands by the presbytery. Take
        pains with these things; be absorbed in them, so that your
        progress will be evident to all. \textbf{Pay close attention to
        yourself and to your teaching; persevere in these things, for as
        you do this you will ensure salvation both for yourself and for
        those who hear you.}

    \end{quote}

    So, before we delve into verse 16 more deeply, I want to just look
    with you at an overview of this entire passage, and I want to point
    out some observations by way of context, and then we'll look at
    Paul's flow of thought leading into verse 16, and then we'll parse
    out verse 16 itself and see what kind of conclusions we can draw
    from it. Sound good? Okay, here we go.

    First, I want to highlight that this entire passage is all about
    doctrine. Paul is focused here on doctrine. And to show you that, I
    want to point out the number of times Paul uses the word, or a root
    of it, for ``doctrine''. First, in verse 6. Paul says, ``In point
    out these things to the brethren, you will be a good servant of
    Christ Jesus, constantly nourished on the words of the faith and of
    the sound \emph{doctrine} which you have been following.'' There it
    is, \emph{didaskalia}, or ``doctrine'', as you see there.

    Now skip down to verse 11. We see that Paul writes, ``Command and
    \emph{teach} these things.'' The word ``teach'' there is the verb
    form of the word ``doctrine''. 

    Now look in verse 13. Paul says to ``devote yourself to the public
    reading of Scripture, to exhortation, to \emph{teaching}.'' There's
    the word for ``doctrine'' again. We could have just as easily
    translated this, ``devote yourself to the public reading of
    Scripture, to exhortation, to \emph{doctrine}.''

    And then finally, in our verse tonight, verse 16, we read, ``Keep a
    close watch on yourself and on the \emph{teaching}.'' And again,
    that's the word \emph{didaskalia}, or ``doctrine''. 

    So, four times in this section Paul refers to ''doctrine''. He
    starts this section to Timothy by emphasizing doctrine, he
    continually pushes Timothy to focus on doctrine, even commanding it,
    and then he closes this section by emphasizing doctrine. There's no
    question that in Paul's mind, doctrine was paramount.

    The second thing I want to look at is the words that Paul uses to
    emphasize doctrine. They are strong words. Words of command. In
    verse 6, Timothy is to be trained in doctrine. In verse 13, Timothy
    is to be devoted to doctrine. In verse 15, Timothy is to be immersed
    in doctrine. And in verse 16, Timothy is ``keep a close watch'' on
    doctrine and persist in it.  Let's look at each of those words a
    little more fully.

    In verse 6, Paul commands Timothy to be \emph{trained} in doctrine.
    The word Paul uses is \emph{entrepho}. It means literally ``to be
    nourished''. The root words means ``feed''. Now, what's the purpose
    of food? It's to sustain life. When we eat physical food, we are
    taking in the nutrients and the vitamins and the energy our bodies
    need to continue function as God designed them, so that we may use
    our bodies to accomplish that which God wills. But where does our
    spiritual life come from? How do we take in the nutrients into our
    spiritual life so that we are equipped to be spiritually healthy
    followers of Christ? That happens by being taught doctrine. In other
    words, if you're not ingesting sound, Biblical doctrine on a regular
    basis, you are a malnourished and spiritually anemic individual.

\end{document}

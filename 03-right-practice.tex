\documentclass[letterpaper, 12pt]{article}

\usepackage[margin=1in]{geometry}

\begin{document}

    \thispagestyle{empty}

    \begin{center}

        \textbf{\large{What To Look For In A Church: Right Practice}}

    \end{center}

    I'm thankful to have this opportunity again this evening as we wrap
    our study on What To Look For In A Church. Of course, we're
    competing with the Super Bowl, so for those that remain I trust that
    it will be even more of a blessing to you tonight. 

    I want to review a little bit about what we've covered so far to
    make sure we're up to speed on things. We're asking the question,
    ``What should I look for in a church?'' And I noted that we have
    several new members as well as regular attenders, and it's always a
    difficult question to sort through in your brain what should be the
    defining characteristics of a good church. So hopefully you can use
    this series to either confirm or guide your decisions, and we would
    love to see you decide to become a member if you are not already.
    But more than that, I hope you realize that most churches do not
    have these characteristics, and there are many folks in those
    churches who just want something better, but they don't how or where
    to go. So maybe you can use this series as a launching pad to reach
    out to those people and invite them here so they can be part of a
    Biblical church community.

    Recall that I suggested three areas to look for in a church: right
    leadership, right doctrine, and right practice. In the first week,
    we saw the necessity of right leadership, where a local church is
    shepherded by a plurality of qualified men who fulfill Peter's
    command in 1 Peter 5 to ``shepherd the flock of God among you.''
    There were two key points that I want you take away from that:
    first, that shepherds are to be examples of the flock. Remember that
    they are to be a ``type'' of Christ, or a die that has been stamped
    by the image of Christ through repetition. Second, the way that
    pastors are to be that example is through continual and repetitive
    expository preaching and teaching. As the pastor devotes himself to
    the study and exposition of Scripture, that impacts his life, and
    then he communicates that study to us by ``preach[ing] the word'',
    and then is able to show us what that looks like because he's
    already wrestled with the text and come to grips with it in his own
    life.

    So that was the crux of Right Leadership. Then, last week, we looked
    at Right Doctrine. And we studied through 1 Timothy 4 where Paul
    commands Timothy to ``Pay close attention to yourself and the
    teaching.'' We saw that doctrinal preaching and teaching in the
    church, by gifted and qualified pastors and other teachers, is of
    paramount importance. And that stands in sharp contrast to most
    churches today, where most so-called ``sermons'' are just feel-good
    pep-talks that designed to puff up your ego or make you feel better
    about yourself, instead of drilling biblical, doctrinal truth into
    your brain so that the truth of God's Word is deposited in your
    heart and then is lived out in your life.

    Finally, tonight, we're going to round out our series by looking at
    the third aspect of a good church, and that is Right Practice. What
    does a church do, or what should a church look like in its corporate
    life? That's the question we want to look at tonight, and we're
    going to do that by looking in Hebrew 10.

    Now, the question of Right Practice of a church has been a widely
    debated question with many answers throughout church history. The
    Reformers answered the question in this way: they identified three
    markers of right practice within the church. First, they said
    ``Wherever the gospel is preached rightly.'' They used that as more
    of a blanket statement along the lines of what we've covered the
    last two weeks, meaning qualified men preaching sound doctrine. So
    that was the first marker for the Reformers. 

    Second, they said that
    a church should be the place where ``The Sacraments are rightly
    administered.'' Many Reformers viewed communion and baptism as what
    we call ``means of grace'', and to delve into what they meant by
    that is a little beyond us tonight, but in terms of a good church,
    what they meant at a minimum was that communion and baptism were
    only administered to those within the church that were in good
    standing. So, as far as the visible church goes, communion and
    baptism were the means by which churches made the public statement
    to the world that because of Christ's blood on the cross, here is
    who partakes of that. 

    Third, and related to those ordinances, was
    ``The right exercise of church discipline.'' For the Reformers,
    excluding unrepentant sinners from communion and witholding baptism
    from them was the most visible sign of church discipline. We see
    the Lord's pattern for church discipline in Matthew 18. Church
    discipline is an often misunderstood topic. Some think that's a tool
    for the pastor to keep the folk in line; some think it's a tool for
    retribution against someone you don't like; mostly today people
    aren't even aware of church discipline or what it is supposed to be.
    But reading through Matthew 18, 1 Corinthians 5, and 2 Corinthians
    2, it's clear that church discipline has a Biblical, twofold
    purpose: first, it's goal is to restore a wayward sinner to
    fellowship within the church. It's not a punishment, it's a rescue
    mission. Second, it's designed to maintain the purity of the church.
    Unchecked sin running rampant within a church will destroy that
    church, and so church discipline functions as the corporate body's
    immune system to regulate and expel the leaven of sin.

    So those were the three markers of a true church
    according the Reformers: pure preaching of the gospel, pure
    adminstration of the sacraments, and the exercise of church
    discipline. Those are good markers, and I encourage you to research
    some of the historical confessions and the Reformers' writings if
    you want to find out more about those three items.

    More recently, however, others have answered the same question
    differently. Phil Johnson has given a sermon in the last few years
    where he takes those same three points that the Reformers gave, and
    he summarized them by expositing through Revelation 2, where Jesus
    scolds the church at Ephesus because they had ``left your first
    love.'' And Phil Johnson concludes that if you had to identify just
    a single characteristic that encapsulates everything a church should
    be, it should be their love for Christ. And so he goes into more
    detail on that, and I encourage to listen to that sermon as well.
    You should be able to search for ``phil johnson love for christ'' or
    ``phil johnson revelation 2'' or something like that and you should
    be able to find it. If you can't, and you're interested, let me know
    and I'll find it and send it to you. It's well worth it.

    So those are just some historical and recent ideas that others have
    had about what a church should look like. And tonight I want to take
    a different approach, and I want to answer that question by
    expositing through Heb. 10:19--25. You can turn there with me, and
    as you are finding that passage, we're going to see what some of the
    fundamental characteristics of corporate church life should look
    like according to God's Word. So Hebrews 10, starting in verse 19.
    The author writes,

    \begin{quote}

        Therefore, brethren, since we have confidence to enter the holy
        place by the blood of Jesus, by a new and living way which He
        inaugurated for us through the veil, that is, His flesh, and
        since we have a great priest over the house of God, let us draw
        near with a sincere heart in full assurance of faith, having our
        hearts sprinkled clean from an evil conscience and our bodies
        washed with pure water. Let us hold fast the confession of our
        hope without wavering, for He who promised is faithful; and let
        us consider how to stimulate one another to love and good deeds,
        not forsaking our own assembling together, as is the habit of
        some, but encouraging one another; and all the more as you see
        the day drawing near.

    \end{quote}

    Before we get into these verses in detail, I want to walk you
    through the context leading up to these verses so you can pick up
    the train of thought that the writer of Hebrews has here in this
    passage. These verse in Hebrews 10 are really the end of a long
    section that really began back at the end of chapter 4. So turn with
    me there back to Heb. 4:14--16. Last week we saw Brett preach on
    Sunday morning on Jesus sympathizing with our weakness. That is
    really the introductory thought that the writer of Hebrews uses to
    build the idea that Jesus is a better and greater high priest. You
    see in verse 14 that the writer identifies Jesus as a ``great high
    priest''.  

    Then, starting in chapter 5, the author then shows how Jesus is a
    priest that comes from a greater spiritual office and lineage than
    did Aaron. Aaron was a Levite, and under the Mosaic covenant the
    priesthood rested with Aaron and his sons. However, there was a
    priesthood before Aaron, and that is why the writer identifies Jesus
    as a ``high priest according to the order of Melchizedek''
    throughout chapter 5 and chapter 6.

    Then in chapter 7, we see why Jesus, as a priest according to the
    priesthood of Melchizedek, is so much better: he appeals to a kind
    of ``federal headship'', and in effect says that because Abraham was
    the father of the Levite tribe from which we have Aaron and the
    Levite priests, then because Abraham paid tithes to Melchizedek,
    then in effect Aaron and the Levites also paid tithes to
    Melchizedek, making Melchizedek a superior priesthood to Aaron's
    priesthood. He then moves to argue that because Aaron's priesthood could never
    attain perfection by itself, there was a requirement for a better
    priest to arise from the greater priesthood, the priesthood of
    Melchizedek. And so the question in chapter 7, is ``Who is that
    priest?''

    We find that out in Heb. 8:1. Look with me at that real quick. We
    see a plain statement about the author's thesis when he writes,
    ``Now the main point in what has been said is this: we have such a
    high priest.'' In other words, the author identifies Jesus as the
    superior high priest that comes from the priesthood of Melchizedek.
    Then, throughout chapter 8, chapter 9, and all the way into chapter
    10, he then shows why Jesus, as the high priest from the priestly
    line of Melchizedek, is so much better.

    He starts by reminding us of the New Covenant in Jeremiah 33, then
    in chapter 9 he highlights the shortcomings of the Old Covenant, the
    Mosaic Law, as being unable to truly take away sins, and then he
    points to Jesus as the mediator of the New Covenant, which is
    better, because through instituting the New Covenant, Jesus did what
    the Levite priests with all their sacrifices could never do:
    permanently and forever take away our sins. Jesus became the
    once-and-for-all sacrifice that fulfilled the righteous requirement
    of the Law to atone for sin, and because He has provided forgiveness
    for us, there no longer remains a need for continual offerings and
    sacrifices. That's the point in Hebrews 10, up through verse 18, and
    that brings us to verses 19--25 and our passage tonight.

    Now, you might be looking at these chapters, and saying to yourself,
    ``Well that's a lot of ground you just covered! How do you know this
    is all a single, related section?'' Well, I'm glad you asked. Let me
    show you. Look in verses 19--25 of chapter 10, and I want you to
    notice two phrases that occur in a specific order: first in verse
    22, we see ``let us draw near with a sincere heart''. Then, in the
    next verse, verse 23, we see the phrase ``Let us hold fast the
    confession of our hope''. Do you see those? Got it?

    Ok, flip with me back to chapter 4. Look in verses 14--16 again, and
    what do we see there? We see first in verse 14 a similar phrase,
    ``let us hold fast our confession'', and then in verse 16, we see
    again, ``let us draw near with confidence''. And notice that those
    two statements are in the reverse order from how we see them in
    chapter 10. In other words, Hebrews 4:14--16 and Hebrews 10:19--25
    are acting like kind of bookends that bracket this central flow of
    thought in the chapters in between. In fact, we even have more
    detail as to the connected nature of these several chapters, because
    we've already looked at 8:1, where we are told what the main point
    of the writer is, and that occurs approximately in the middle. So
    there is clearly an intent by the writer to structure these chapters
    in such a way as to make a coordinated argument for a central point,
    and we're told what that central point is in chapter 8, and our
    passage tonight represents the part where he wraps up that thought
    and finishes it.

    In other words, our passage tonight represents the conclusions based
    on that thought. Now, recall what that main point is. The main point
    of this whole big section from chapter 4 all the way through to
    chapter 10 is that Jesus is that high priest. He mediates a new and
    better covenant because He Himself became a new and better and final
    sacrifice to atone for the sins of His people. And now, starting in
    verse 19 of chapter 10, we see the writer begin to connect the dots
    from that great truth to what we're going to look at tonight. Look
    at the beginning of verse 19. We see the word ``Therefore''. Of
    course you've heard the little jingle about Bible study that
    whenever you see the word ``therefore'', you should look to see why
    is it there for. I like how one dictionary translates this word:
    ``here's how the dots connect.'' 

    Notice how the writer connects the dots for us: as a result of Jesus
    being that great high priest who has once and for all atoned for
    sins because of His sacrifice, because of all that, we have
    confidence to enter the holy place. In the ancient Hebrew
    tabernacle, and then later in the temple, there was a special room
    in the center where the Ark of the Covenant lay. It was the place
    where the very presence of God dwelt under the Mosaic covenant. It
    was separated from the rest of the tabernacle and temple by a thick
    veil.

    It was called The Most Holy Place, or The Holy of Holies. And no one
    was allowed in except for one time per year, the high priest of all
    Israel would cleanse himself, don his priestly garments, and enter
    the Most Holy Place to make sacrifices to Yahweh on behalf of
    himself and the entire nation. You and I couldn't go in; only the
    high priest could enter, and only once per year.

    But now, look in verse 20: Jesus, as the Great High Priest of new
    and better covenant, has opened for us a ``new and living way''. He
    ``inaugurated'', or instituted something that was novel and has not
    been seen before. And notice the path this new way takes: it goes
    straight through the veil. Recall that Matthew records in Matt.
    27:51 that Jesus ``yielded up His spirit. And behold, the veil of
    the temple was torn in two from top to bottom.'' In other words,
    when Jesus died, God supernaturally tore the veil separating Himself
    from His people so that all who belong to Him now have unfettered
    access. 

    But notice that this new way isn't just the absence of the veil. The
    veil has been replaced by something new; something different; it has
    been replaced not by a something, but by a someone. The writer says
    this new way through the veil is through ``His flesh''. Recall what
    Jesus told His disciples in Jn. 14:6, ``I am the way, and the truth,
    and the life; no one comes to the Father but through Me.''

    This new and living way is not through the ritual sacrifices and
    observances of the Old Covenant, the Mosaic Law, in which you could
    only come to God by proxy through the ministry of the high priest,
    but instead all of God's people have access to the Father through
    faith in Jesus Christ. This, verses 19-20, is a powerful, powerful
    statement on the Doctrine of Justification by Faith. Our
    relationship with God is no longer by proxy; but it is through
    simple faith in Christ Alone that we all have access into the Holy
    of Holies.

    But notice that the basis for this access is not \emph{just} based
    on Christ's sacrificial death on the cross. Look in verse 21: ``and
    since \emph{we have} a great priest over the house of God.'' Jesus
    didn't offer Himself as a once and for all sacrifice, fulfill the
    righteous requirement of the Law on our behalf, and then after He
    died and was buried, and that was it. No, what happened after that
    three days later? He rose again! He was resurrected and lives
    forevermore! So He continues to be our ``great priest over the house
    of God''. In fact, look back in Heb. 7:25: ``Therefore He [that is,
    Jesus] is able also to save forever those who draw near to God
    through Him, since He always lives to make intercession for them.''
    Our Great High Priest not only offered Himself as a perfect and
    final sacrifice for our sins so that we might have unfettered access
    to the Father, but He also rose again and continues as our Great
    High Priest because He Himself is the way to the Father, and He
    continues His ministry as our Great High Priest by interceding on
    our behalf.

    So these two doctrines, the Doctrines of Justification by Faith and
    the Doctrine of the Resurrection, these are core, powerful doctrines
    that the writer uses to establish the basis for our three points
    tonight. So back in Hebrews 10, let's look at verses 22--25, and
    we're going to see three points that the writer extracts from these
    two powerful doctrines. What does the right practice of a church
    look like? The writer answers that based on the reality of faith in
    Jesus Christ, His death, burial, and resurrection, the right
    practice of a church looks like this: Be Saved, Be Sanctified, and
    Be Serving. And so we're going to look at each of those in turn.

    Our first point tonight: Be Saved. Look in verse 22. Because we now
    have free access ``to enter the holy place by the blood of Jesus'',
    we are to ``draw near with a sincere heart in full assurance of
    faith, having our hearts sprinkled clean from an evil conscience and
    our bodies washed with pure water.''

    We are to ``draw near''. Literally, we are to ``go towards''. We,
    who were far away, are now to come to the holy place, the very
    presence of God, so that we are now close beside. But notice there
    is a particular manner in which we are to draw near. We are to have
    ``a sincere heart in full assurance of faith.'' 

    The idea of a ``sincere heart'' is something that's true.
    Something that's genuine. Something that reflects reality. And what
    is it that is characterize this true reflection of reality. What is
    it that is supposed to be genuine? It's a ``full assurance of
    faith''. 

    That phrase, ``full assurance of faith'', is usually translated just
    like that: ``full assurance''. But there are other occurrences in
    Scripture where it is also translated as ``accomplished''. One
    dictionary defines this word as ``bearing the work of God to the
    fullest extent''.

    But taking it just as it, let's summarize so far and see where we're
    at with this verse: because of Christ's work as our Great High
    Priest, we are to go towards and be near Him, and we are to do so
    because we have a genuine heart that reflects the reality of our
    full assurance of faith.

    Now, what is the characteristic of our ``full assurance of faith''.
    Well, the writer tells us, ``having our hearts sprinkled clean from
    an evil conscience and our bodies washed with pure water.'' The idea
    of our hearts being sprinkled clean and our bodies being washed is
    the idea of a ceremonial cleansing. It recalls the high priest, as
    part of the ritual sacrifices in the Most Holy Place, would sprinkle
    blood as a symbol of being covered with Christ's blood. 

    Look with me in Leviticus 15. Leviticus 15 is where we find the
    instructions for observing the Day of Atonement, which was the one
    day of the year where the high priest would enter the Most Holy
    Place and offer sacrifices for himself and for the entire nation of
    Israel. Look in verse 4: the high priest was to first bathe his body
    in water and then put on the ceremonial garments of his office. Then
    in verse 14, we read, ``he shall take some of the blood of the bull
    and sprinkle it with his finger on the mercy seat on the east side;
    also in front of the mercy seat he shall sprinkle some of the blood
    with his finger seven times.'' Then again in verse 19, we see that
    the high priest is to sprinkle blood, but this time it says the
    purpose is to ``cleanse it, and from the impurities of the sons of
    Israel consecrate it.''

    In other words, back in Hebrews 10, where the writer says that our
    hearts are to be sprinkled clean and our bodies washed, this is a
    reference to our inmost being, our self, being consecrated unto God
    and therefore having our hearts ``sprinkled clean'' from the
    defilement of our sin nature and as a result our outward lives
    reflect that inward ``cleaning'' by being ``washed''.

    In other words, verse 22 is talking about being regenerated unto
    saving faith! This points to the first right practice of a church,
    and that is regenerate church membership. In order for a church to
    function properly, we first have to recognize that a church is made
    up of only believers. Today, unfortunately, the practice of churches
    to have regenerate church membership seems to have fallen by the
    wayside, and most churches now correctly deduce that our mission is
    to reach the lost, but they do so by bringing the lost into the
    church, and thinking that by ``meeting people where they're at''
    that the lost will be more receptive to the gospel, and therefore be
    converted.

    And the problem with that is that it doesn't work. What happens in
    those instances is that the lost aren't converted, and the church
    then becomes populated with unbelievers, and eventually begins to
    look more like the world than the church. And the reason for that
    goes back to our text tonight: they didn't observe the first
    right practice of a church. Be saved!

    Let's look at our second point now: Be Sanctified. Look with me in
    verse 23. The author says, ``Let us hold fast the confession of our
    hope without wavering.'' This idea of ``holding fast'' is an
    interesting word. Paul uses this word in his letter to Philemon.
    Look with me in Philemon quickly, starting at verse 12. Recall that
    Paul was in prison in Rome, and a runaway slave from Colosse, named
    Onesimus, had found his way to Rome, come into contact with Paul,
    was converted by the gospel, and now Paul was sending him back to
    his master, Philemon, in Colosse. 

    Now, Paul didn't want to send Onesimus away, but he was doing so
    because it was the right thing for Onesimus to do. Look in verse 12,
    where Paul writes, ``I have sent him back to you in person, that is
    sending my very heart, \emph{whom I wished to keep with me}.'' That
    phrase, ``keep with me'', that's the same word as our text that says
    we are to ``hold fast''. Paul wanted Onesimus to stay and minister
    with him. He wanted Onesimus to continue on with Paul in Rome and be
    his partner in ministry.

    And that's the idea in Hebrews 10:23 of ``holding fast''. We are to
    continue in our confession of hope. Other synonyms include ``keeping
    in memory'', stay, holding firmly, or occupy. It's the idea of
    persistence or continuity. It is to be a continual and perpetual
    thing.

    Now, notice that we are to hold fast in a certain way: we are to do
    so ``without wavering''. The Greek word for ``without wavering'' is
    literally ``unbending''. Our holding fast is to be characterized by
    being unyielding or unbending with a steadfast, immovable conviction
    towards the confession of our hope. Paul puts it this way in 1 Cor.
    15:58. He writes, ``Therefore, my beloved brethern, be steadfast,
    immovable, always abounding in the work of the Lord, knowing that
    your toil is not in vain in the Lord.''

    So that's the idea of ``without wavering''. Lets recap quickly: we are to hold fast
    the confession of our hope, and we are to do so without bending.
    Now, what is this ``confession of our hope''. Well, as you may know,
    in the Bible when you see the word ``confession'', that generally
    means ``agreement''. It's literally translated, ``to say the same
    thing about''. It takes on the idea of a profession that is based on a firm conviction. 

    That conviction is ``our hope''. Today what we call ``hope'' is
    really just something that we wish for, and maybe or maybe not it
    will happen. Usually it doesn't. But that's not how the Bible talks
    about hope. Hope in the Bible is always derived from expectation of
    something that will happen. It's a certainty. A guarantee. 

    What is that we confess that is a guaranteed certainty? Well, in the
    context of Hebrews, I believe the hope that the writer is referring
    to is that because of Christ's work as our Great High Priest, we now
    have confident access to draw near to God \emph{and stay there}. In
    other words, I believe this hope to be referring to the blessing of
    spending eternity in the presence of God forever in Heaven. That's
    our hope as Christians; Christians aren't just saved and then we are
    left alone to be by ourselves. No, we are saved so that we might
    have a right relationship with God for all eternity. As Rev. 21:3
    puts it: ``Behold, the tabernacle of God is among men, and He will
    dwell among them, and they shall be His people, and God Himself will
    be among them.'' That's our hope.

    And because of that hope, we are to ``hold fast'' ``without
    wavering''. In other words, we are to persevere. Look with me 2
    Pet. 1:1--8. Peter puts it this way. He writes,

    \begin{quote}

        Simon Peter, a bond-servant and apostle of Jesus Christ, To
        those who have received a faith of the same kind as ours, by the
        righteousness of our God and Savior, Jesus Christ: Grace and
        peace be multiplied to you in the knowledge of God and of Jesus
        our Lord; seeing that His divine power has granted to us
        everything pertaining to life and godliness, through the true
        knowledge of Him who called us by His own glory and excellence.
        For by these He has granted to us His precious and magnificent
        promises, so that by them you may become partakers of the divine
        nature, having escaped the corruption that is in the world by
        lust. Now for this very reason also, applying all diligence, in
        your faith supply moral excellence, and in your moral
        excellence, knowledge, and in your knowledge, self-control, and
        in your self-control, perseverance, and in your perseverance,
        godliness, and in your godliness, brotherly kindness, and in
        your brotherly kindness, love. For if these qualities are yours
        and are increasing, they render you neither useless nor
        unfruitful in the true knowledge of our Lord Jesus Christ.

    \end{quote}

    Here we see that perseverance is one of the things that we are
    exercise ourselves in, but notice that perseverance is tied to our
    sanctification. Peter is saying that because of the Imputed
    Righteousness of Christ, He has granted to us everything we need for
    life and Godliness, and that happens through knowledge of the
    Scriptures. And those Scriptures contain ``precious and magnficent
    promises'', i.e., hope for us, that make us partakers of Christ. And
    because of all that, we are to increase in these qualities. And
    among those qualities are perseverance.

    In other words, back in Hebrews 10, our unyielding perseverance in
    the confession of our hope is a marker of our increasing
    sanctification as we are conformed to the image of Christ. And so
    the second marker of the right practice of a church is to Be
    Sanctified!

    Finally, our third point of a church's right practice: Be Serving.
    Look with me in verses 24--25 of Hebrews 10. ``Let us consider how
    to stimulate one another to love and good deeds, not forsaking our
    own assembling together, as is the habit of some, but encouraging
    one another; and all the more as you see the day drawing near.''

    Oh boy, it's about to get real with this verse. First, let's look at
    this word ``consider''. The Greek word is \emph{katanoeo}. It's
    translated elsewhere in Scripture as ``observing'' or ``detecting''.
    It means literally, ``to reason from up to down''. In other words,
    this is talking about deductive reasoning. We are to observe others
    in the church, detect something about them, and then do something
    about it.

    Now, what is it we are supposed to ``consider''? We are to consider
    ``how to stimulate one another''. This is an interesting word as
    well. It's the same word used between Paul and Barnabas in Acts 15
    where it says that ``sharp disagreement'' arose over John Mark. The
    Greek word is \emph{paroxusmos}. It's where we get our word
    ``paroxysm''. It's a conniption fit.

    Now, this isn't saying go be a jerk to people and push their buttons
    until they just go off on you in an outburst of anger. The idea is
    that we are to consider the spiritual needs of others and then
    figure out a way to meet those needs in such a way that the person
    has a clear choice: they can either engage in service and rise to
    the occasion, or they can consciously sit on the sidelines.

    Now look, the things we are to ``stimulate'' others towards are
    ``love and good deeds''. These are two related things. The one goes
    with the other. We are to consider how to help others grow in their
    love for Christ and love for the brethren, and as a result they are
    to grow in their good works ``which God prepared beforehand so that
    we would walk in them.'' That's Eph. 2:10.

    Now, how do we do all that? What is required for us to be able to
    deductively reason how we can help each other grow in love and good
    deeds, such that we have no option but to respond one way or
    another? What's the minimum requirement for that to happen?

    Verse 25 tells us: ``not forsaking our own assembling together''.
    This should make sense to us, right? If we call ourselves
    Christians, we are to ``consider how to stimulate one another to
    love and good deeds''. And you can't do that if you have forsaken
    the church, the very people you are supposed to stimulate.

    And oh boy, do we have a problem with this today. Over the last 2+
    years we have seen a totalitarian takeover of society, using a
    so-called ``pandemic'' with over a 99.9\%+ survival rate as an
    excuse, to shut down every good thing about our lives that God has
    blessed us with. There have been lies after lies about statistics.
    Effective treatments have intentionally been discourage and even
    prohibited, and knowingly ineffective and even dangerous treatments
    have been promoted. Churches were closed as ``super-spreaders'' while bars and abortion
    clinics remained open as ``essential services''. \\

    And churches have gone along with it! \\

    There are still churches today that are either just starting to meet
    in person, or are still conducting ``services'' online. And it's
    even worse. Now we have a suspect vaccine that has numerous
    concerns, known side effects, including death, that have not been
    addressed, and heavy-handed, illegal mandates to force it upon
    people whether they need it or not. And there are churches that are
    playing along with that evil game, requiring these things to be able
    to worship God.  Where God has given us free access to the Holy of
    Holies through Jesus Christ, these hirelings have recreated the veil
    that separates us from God, only this time with stupid mask
    restrictions and overbearing vaccine mandates in order to be able to
    draw near and worship God.

    It's not only wrong-headed, it's evil. And if you think this is all
    just a coincidence, that those who rule over us have our best
    interest at heart, then you are being either ignorantly naive or
    intentionally complicit. And by the way, the government just this
    week released an official memo that labels what I just said as a
    terrorist threat. I, who have nearly 300 combat hours in an F-15E in
    defense of our freedoms, am now a ``terrorist threat''.

\end{document}

\documentclass[letterpaper, 12pt]{article}

\usepackage[margin=1in]{geometry}

\begin{document}

    \thispagestyle{empty}

    \begin{center}

        \textbf{\large{What To Look For In A Church: Right Practice}}

    \end{center}

    I'm thankful to have this opportunity again this evening as we wrap
    our study on What To Look For In A Church. Of course, we're
    competing with the Super Bowl, so for those that remain I trust that
    it will be even more of a blessing to you tonight. 

    I want to review a little bit about what we've covered so far to
    make sure we're up to speed on things. We're asking the question,
    ``What should I look for in a church?'' And I noted that we have
    several new members as well as regular attenders, and it's always a
    difficult question to sort through in your brain what should be the
    defining characteristics of a good church. So hopefully you can use
    this series to either confirm or guide your decisions, and we would
    love to see you decide to become a member if you are not already.
    But more than that, I hope you realize that most churches do not
    have these characteristics, and there are many folks in those
    churches who just want something better, but they don't how or where
    to go. So maybe you can use this series as a launching pad to reach
    out to those people and invite them here so they can be part of a
    Biblical church community.

    Recall that I suggested three areas to look for in a church: right
    leadership, right doctrine, and right practice. In the first week,
    we saw the necessity of right leadership, where a local church is
    shepherded by a plurality of qualified men who fulfill Peter's
    command in 1 Peter 5 to ``shepherd the flock of God among you.''
    There were two key points that I want you take away from that:
    first, that shepherds are to be examples of the flock. Remember that
    they are to be a ``type'' of Christ, or a die that has been stamped
    by the image of Christ through repetition. Second, the way that
    pastors are to be that example is through continual and repetitive
    expository preaching and teaching. As the pastor devotes himself to
    the study and exposition of Scripture, that impacts his life, and
    then he communicates that study to us by ``preach[ing] the word'',
    and then is able to show us what that looks like because he's
    already wrestled with the text and come to grips with it in his own
    life.

    So that was the crux of Right Leadership. Then, last week, we looked
    at Right Doctrine. And we studied through 1 Timothy 4 where Paul
    commands Timothy to ``Pay close attention to yourself and the
    teaching.'' We saw that doctrinal preaching and teaching in the
    church, by gifted and qualified pastors and other teachers, is of
    paramount importance. And that stands in sharp contrast to most
    churches today, where most so-called ``sermons'' are just feel-good
    pep-talks that designed to puff up your ego or make you feel better
    about yourself, instead of drilling biblical, doctrinal truth into
    your brain so that the truth of God's Word is deposited in your
    heart and then is lived out in your life.

    Finally, tonight, we're going to round out our series by looking at
    the third aspect of a good church, and that is Right Practice. What
    does a church do, or what should a church look like in its corporate
    life? That's the question we want to look at tonight, and we're
    going to do that by looking in Hebrew 10.

    Now, the question of Right Practice of a church has been a widely
    debated question with many answers throughout church history. The
    Reformers answered the question in this way: they identified three
    markers of right practice within the church. First, they said
    ``Wherever the gospel is preached rightly.'' They used that as more
    of a blanket statement along the lines of what we've covered the
    last two weeks, meaning qualified men preaching sound doctrine. So
    that was the first marker for the Reformers. 

    Second, they said that
    a church should be the place where ``The Sacraments are rightly
    administered.'' Many Reformers viewed communion and baptism as what
    we call ``means of grace'', and to delve into what they meant by
    that is a little beyond us tonight, but in terms of a good church,
    what they meant at a minimum was that communion and baptism were
    only administered to those within the church that were in good
    standing. So, as far as the visible church goes, communion and
    baptism were the means by which churches made the public statement
    to the world that because of Christ's blood on the cross, here is
    who partakes of that. 

    Third, and related to those ordinances, was
    ``The right exercise of church discipline.'' For the Reformers,
    excluding unrepentant sinners from communion and witholding baptism
    from them was the most visible sign of church discipline. We see
    the Lord's pattern for church discipline in Matthew 18. Church
    discipline is an often misunderstood topic. Some think that's a tool
    for the pastor to keep the folk in line; some think it's a tool for
    retribution against someone you don't like; mostly today people
    aren't even aware of church discipline or what it is supposed to be.
    But reading through Matthew 18, 1 Corinthians 5, and 2 Corinthians
    2, it's clear that church discipline has a Biblical, twofold
    purpose: first, it's goal is to restore a wayward sinner to
    fellowship within the church. It's not a punishment, it's a rescue
    mission. Second, it's designed to maintain the purity of the church.
    Unchecked sin running rampant within a church will destroy that
    church, and so church discipline functions as the corporate body's
    immune system to regulate and expel the leaven of sin.

    So those were the three markers of a true church
    according the Reformers: pure preaching of the gospel, pure
    adminstration of the sacraments, and the exercise of church
    discipline. Those are good markers, and I encourage you to research
    some of the historical confessions and the Reformers' writings if
    you want to find out more about those three items.

    More recently, however, others have answered the same question
    differently. Phil Johnson has given a sermon in the last few years
    where he takes those same three points that the Reformers gave, and
    he summarized them by expositing through Revelation 2, where Jesus
    scolds the church at Ephesus because they had ``left your first
    love.'' And Phil Johnson concludes that if you had to identify just
    a single characteristic that encapsulates everything a church should
    be, it should be their love for Christ. And so he goes into more
    detail on that, and I encourage to listen to that sermon as well.
    You should be able to search for ``phil johnson love for christ'' or
    ``phil johnson revelation 2'' or something like that and you should
    be able to find it. If you can't, and you're interested, let me know
    and I'll find it and send it to you. It's well worth it.

    So those are just some historical and recent ideas that others have
    had about what a church should look like. And tonight I want to take
    a different approach, and I want to answer that question by
    expositing through Heb. 10:19--25. You can turn there with me, and
    as you are finding that passage, we're going to see what some of the
    fundamental characteristics of corporate church life should look
    like according to God's Word. So Hebrews 10, starting in verse 19.
    The author writes,

    \begin{quote}

        Therefore, brethren, since we have confidence to enter the holy
        place by the blood of Jesus, by a new and living way which He
        inaugurated for us through the veil, that is, His flesh, and
        since we have a great priest over the house of God, let us draw
        near with a sincere heart in full assurance of faith, having our
        hearts sprinkled clean from an evil conscience and our bodies
        washed with pure water. Let us hold fast the confession of our
        hope without wavering, for He who promised is faithful; and let
        us consider how to stimulate one another to love and good deeds,
        not forsaking our own assembling together, as is the habit of
        some, but encouraging one another; and all the more as you see
        the day drawing near.

    \end{quote}

\end{document}
